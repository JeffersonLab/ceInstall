\ifx\pdfoutput\undefined      % compilazione htlatex
\documentclass{article}
\DeclareGraphicsExtensions{.png, .gif, .jpg}
\newcommand{\href}[2]{\Link[#1]{}{} #2 \EndLink}
\newcommand{\hypertarget}[2]{\Link[]{}{#1} #2 \EndLink}
\newcommand{\hyperlink}[2]{\Link[]{#1}{} #2 \EndLink}
\else                         % compilazione pdflatex
\documentclass{article}
\usepackage{graphicx}
\usepackage{listings}
\usepackage{fancyhdr}
\usepackage{wrapfig}
\usepackage{multirow}
\usepackage{lscape}
\usepackage{amssymb,amsmath}
\pdfpagewidth 8.5in
\pdfpageheight 11in
\setlength\textwidth{6.5in}
\setlength\textheight{8.1in}
\setlength\oddsidemargin{0in}
\setlength\evensidemargin{0in}
\setlength\topmargin{-0.6in}
\setlength\footskip{0.6in}
\setlength\headsep{0.6in}
\usepackage[hyperindex]{hyperref}
\usepackage{qtree}
\newcommand{\percent}{\,^0\!/_0}
 \hypersetup{
    bookmarks=true,         % show bookmarks bar?
    unicode=false,          % non-Latin characters in Acrobat’s bookmarks
    pdftoolbar=true,        % show Acrobat’s toolbar?
    pdfmenubar=true,        % show Acrobat’s menu?
    pdffitwindow=true,      % page fit to window when opened
    pdfauthor={Maurizio},   % author
    pdfsubject={Ungaro},    % subject of the document
    pdfnewwindow=true,      % links in new window
    colorlinks=true,        % false: boxed links; true: colored links
    linkcolor=black,        % color of internal links
    citecolor=blue,         % color of links to bibliography
    filecolor=magenta,      % color of file links
    urlcolor=blue           % color of external links
}
\fi


\begin{document}

\pagestyle{fancy}
\renewcommand{\sectionmark}[1]{\markright{\slshape \thesection\ #1}{}}
% remove the right header because clas12 note will overwrite this 
\fancyhead[R]{} 
\fancyhead[L]{Common Environment for Software Packages}
\fancyfoot[R]{ \sl M. Ungaro, Z. Zhao}
\fancyfoot[L]{ \sl JLAB}

\begin{flushright}
CLAS-NOTE 2014-02\\
{\small svn: https://phys12svn.jlab.org/repos/ceInstall/devel/doc }\\
Version: 1.0
\end{flushright}

\vspace{0.6cm}

\begin{center}
{\Large \bf Common Environment for Software Packages} \hfill \\
\vspace{0.6cm}
{  M. Ungaro, Z. Zhao} \hfill
\end{center}
\vspace{0.6cm}

\abstract{We present a model to organize software packages into releases. 
Shell scripts define the environment variables needed to run and compile vs the packages.
Users can switch between releases and can select different versions of individual packages.}

\tableofcontents


\section{Introduction}
We present a set of tools to define a Common Environment (ce) at Jefferson Lab.
This environment can be used to run and/or compile software against the packages.
The packages supported are:

\begin{itemize}
 \item CLHEP   
 \item XERCESC  
 \item QTDIR   
 \item Geant4  
 \item EVIO    
 \item GEMC    
 \item JANA    
 \item ROOT   
 \item Build   
 \item Banks   
\end{itemize}

\clearpage\newpage
\section{Features}
The packages are organized under a general software release 
number, called JLAB\_VERSION. 
For example JLAB\_VERSION=1.0 sets

\begin{itemize}
 \item CLHEP   to version 2.1.3.1
 \item Geant4  to version 4.9.6.p02
 \item QTDIR   to version 4.8.5
 \item XERCES  to version 3.1.1
 \item ROOT    to version 5.34.13
 \item GEMC    to version 1.0
 \item JANA    to version 0.7.1p3
 \item Build   to version 1.0
 \item EVIO    to version 4.0
 \item Banks   to version 0.9
\end{itemize}
The main features are:

\begin{itemize}
 \item[-] Users can select JLAB\_VERSIONs.
 \item[-] Users can select packages versions independently.
 \item[-] Users can select custom OS, CPU architecture and compiler version
 \item[-] The environment can be used to run and compile vs all the packages
 \item[-] Users can install the packages using the same structure on their 
 personal laptop/desktop.
 \item[-] The packages are tested and maintained on Mac and Linux flavors.
\end{itemize}

\section{Quick Start}

\subsection{production or development version (JLab)}
There are two special version set identifiers available:
\begin{itemize}
\item {\it JLAB\_VERSION='production'}: points to a "recommended/stable" version
     set (presently this is an alias for JLAB\_VERSION=1.0
\item {\it JLAB\_VERSION='devel'}: points to a release that is still under
     development/testing.
\end{itemize}
 These special version sets will only be updated after posting a
 notification to the jlab-scicomp-briefs@jlab.org mailing list.
Users can select either version with:
\begin{itemize}
\item[-] source /site/12gev\_phys/production.csh
\item[-] source /site/12gev\_phys/devel.csh
\end{itemize}
\subsection{Custom JLAB\_VERSION and/or location}
Users can select the root directory containing all packages, and the JLAB\_VERSION:
\begin{itemize}
\item[1.] point JLAB\_ROOT to the location of the software packages
\item[2.] choose a JLAB\_VERSION
\item[3.] source the script that sets the environment
\end{itemize}
with these commands (a JLAB specific example is shown):

\begin{verbatim}
  setenv JLAB_ROOT /site/12gev_phys
  setenv JLAB_VERSION 1.0
  source $JLAB_ROOT/$JLAB_VERSION/ce/jlab.csh
\end{verbatim}
Available JLAB\_VERSIONs can be seen at 
\href{https://data.jlab.org/drupal/?q=node/55}{\it latest JLAB\_VERSIONs}.

\subsection{Custom version of a package}
Users can select a custom version of a package by setting the 
environment variable NAME\_VERSION before sourcing jlab.csh, where NAME is the package name capitalized (ROOT, CLHEP, QT, etc). 
For example to choose geant4 version 9.6.p03 instead of the default 9.6.p02 users can do this:
\begin{verbatim}
  setenv JLAB_ROOT /site/12gev_phys
  setenv JLAB_VERSION devel
  setenv GEANT4_VERSION 4.9.6.p03
  source $JLAB_ROOT/$JLAB_VERSION/ce/jlab.csh
\end{verbatim}
Any user defined NAME\_VERSION will overwrite the default. 

\section{The directory structure}
The software is placed into versions in \$JLAB\_ROOT.
At JLab, \$JLAB\_ROOT is /site/12gev\_phys.
Inside each version there are:
\begin{itemize}
\item the ``ce'' directory contains the environment
\item the ``install''directory contains the installation scripts
\item the 'scons\_bm'' directory contains the scons script to compile software against the packages
\item the OSRELEASE contains the actual packages packages
\end{itemize}
An illustration of the organization is shown in Fig.~1.
\vspace{1cm}

\qtreepadding=5pt  
\Tree [ .JLAB\_ROOT 
             [.1.0  {\ldots}  ] 
             [.1.1 
                   ce 
                   noarch 
                   scons\_bm 
                   install 
                   [ .OSRELEASE 
                        [.clhep 2.1 ] 
                        [.geant4 9.6.p01 9.6.p02 ]  
                        [.qt 4.8.5 ]  
                   ]                                      
              ]
      ]
\begin{center}
Fig.~1: The ce directory structure.
\end{center}   

\clearpage\newpage

\subsection{The OSRELEASE variable}
Multiple OS, CPU architecture and compiler version are maintained using the OSRELEASE variable.
OSRELEASE is setup automatically (by ce/osrelease.py) however users can overwrite the default by setting
the OSRELEASE environment variable before sourcing jlab.csh.
Typical examples of OSRELEASE are shown below:
\begin{itemize}
\item Linux\_CentOS6.2-x86\_64-gcc4.4.6
\item Linux\_RHEL6-x86\_64-gcc4.4.7
\item Darwin\_macosx10.8-x86\_64-gcc4.2.1
\item Linux\_Fedora20-x86\_64-gcc4.8.2
\item Linux\_Ubuntu12.04-x86\_64-gcc4.6
\end{itemize}             
             
\subsection{Screen log}
The scripts output on screen contain informations about the JLAB\_ROOT location,
the JLAB\_VERSION, the OSRELEASE and the packages version. Below is a typical output.

\begin{verbatim}
 > Common Environment Version: <production>  (Tue, 25 Feb 2014)
 > Running as ungaro on ifarm1101
 > OS Release:    Linux_CentOS6.2-x86_64-gcc4.4.6
 > JLAB_ROOT set to:     /site/12gev_phys
 > JLAB_SOFTWARE set to: /site/12gev_phys/production/Linux_CentOS6.2-x86_64-gcc4.4.6

 > CLHEP   version: 2.1.3.1
 > XERCES  version: 3.1.1
 > QTDIR   version: 4.8.5
 > Geant4  version: 4.9.6.p02
 > ROOT    version: 5.34.13
 > GEMC    version: 1.8
 > JANA    version: 0.7.1p3
 > Build   version: 1.0
 > EVIO    version: 4.0

\end{verbatim}
       
\clearpage\newpage
          
\section{Installation}
Users can install the supported packages on their laptop/desktop using the installation scripts inside ``install'':
\begin{verbatim}
   cd $JLAB_ROOT/$JLAB_VERSION/install

    ./go_clhep
    ./go_xercesc
    ./go_mysql
    ./go_qt4
    ./go_geant4
    ./go_sconsscript
    ./go_evio
    ./go_gemc
	
    
	Optional: JLab magnetic field, cern ROOT, JLab banks library, JANA
	
    ./go_fields
    ./go_root
    ./go_banks
    ./go_jana
    
    
\end{verbatim}
A complete set of instructions and requirement for installation can be found
\href{https://gemc.jlab.org/gemc/Support/Entries/2011/8/1_Step_by_Step.html}{\it here}.












\end{document}








